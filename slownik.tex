% Podstawowe definicje dla wszystkich dokumentów

\documentclass[11pt]{mwart}
\setlength{\textwidth}{83pt}

\usepackage[OT4,plmath]{polski}
\usepackage{amsmath,amssymb,amsfonts,amsthm,mathtools}
\usepackage{color}
\usepackage{fontspec}
\usepackage{listings,times}

\usepackage{bbm}
\usepackage[colorlinks=true, urlcolor=blue]{hyperref}
\usepackage{url}
\usepackage{graphicx}
\graphicspath{./images/}

\newcommand{\HRule}{\rule{\linewidth}{0.5mm}}

\newcommand{\term}[1]{
  \indent\textbf{#1}
  \vspace{5pt}
}

\usepackage{multicol}

\usepackage{lmodern} \normalfont
%\DeclareFontShape{EU1}{ptm}{bx}{n} { <-> ssub * cmr/bx/n }{}
%\DeclareFontShape{EU1}{ptm}{m}{sc} { <-> ssub * cmr/m/sc }{}

\usepackage{titletoc}

%\titlecontents{section}[3.8em]{}{\contentslabel{2.3em}}{\hspace*{-2.3em}}{\titlerule*[0.25pc]{ .}\contentspage}{}

\newcommand{\titlep}[1] {
  \begin{titlepage}
    \begin{center}
      \textsc{\LARGE Studencka Pracownia Inżynierii Oprogramowania}
      \textsc{\LARGE Instytut Informatyki Uniwersytetu Wrocławskiego}\\[1.5cm]


      \vspace{3cm}

      % Author and supervisor
      \begin{minipage}{\textwidth}
        \begin{center} \Large
          Łukasz \textsc{Czapliński},
          Diana \textsc{Czepirska},
          Artur \textsc{Jarocki}
        \end{center}
      \end{minipage}

      \vspace{0.5cm}



      % Title
      \HRule \\[0.4cm]
      { \Huge \bfseries Running\\free  \\[1cm] }

      \textsc{\Large #1}\\[0.5cm]

      \HRule \\[1.5cm]

      \vspace{1cm}

      \includegraphics[width=0.15\textwidth]{./non-starred.png}~\\[1cm]
      
      \vfill
      

      \vspace{1cm}

      % Bottom of the page
      {\large Wrocław 2013}

    \end{center}
  \end{titlepage}
  \clearpage
}


\newcommand{\dicen}[1]{
\item\textbf{#1}
}

\begin{document}
\titlep{Słownik pojęć}{1.5}

\chist{1.0 & 2013-10-27 & Powstanie dokumentu & Diana Czepirska \\
  1.1 & 2013-11-26 & Poprawki stylistyczne & Diana Czepirska, \\ & & & Łukasz Czapliński\\
	1.2 & 2013-12-10 & Poprawki stylistyczne & Diana Czepirska\\
	1.3 & 2014-01-03 & Poprawki stylistyczne & Diana Czepirska\\
	1.4 & 2014-01-07 & Dodanie nowych terminów & Diana Czepirska\\
	1.5 & 2014-01-22 & Drobne poprawki & Diana Czepirska\\}

\section{Słownik}
\noindent Dokument ten służy wyjaśnieniu nietypowych terminów z dziedziny projektu ,,Iron Coach''.
\begin{multicols}{2}
\begin{description}
% \dicen{pojecie} - definicja
	\dicen{automatyczna pauza (ang. \textit{auto pause})}-- działanie licznika polegające na tym, że poniżej pewnej prędkości zaczyna on działać jak podczas przerwy (tzn. przestaje naliczać czas i dystans) aż do momentu ponownego osiągnięcia większej prędkości.\\
	\dicen{bieg górski (ang. \textit{mountain running})}-- intensywny trening polegający na biegu w terenie, na którym suma przewyższeń stanowi minimum kilka procent (np. 5\%) całej trasy.\\
	\dicen{biegacz (ang. \textit{runner})}-- sportowiec (użytkownik aplikacji), którego interesuje przede wszystkim bieganie. Osoba taka może korzystać ze specjalnego modułu aplikacji przeznaczonego dla biegaczy. \\
	\dicen{energia (ang. \textit{energy})}-- oszacowane zużycie energii (podawane w kilokaloriach) podczas treningu.\\
	\dicen{HIIT (z ang. \textit{High Intensity Interval Training} -- trening interwałowy o~wysokiej intensywności)}-- rodzaj treningu interwałowego, który składa się z krótkich okresów bardzo intensywnego wysiłku przeplatanych krótkimi okresami umiarkowanego wysiłku, przykładowo: 30 sekund sprintu na~przemian z 15 sekundami truchtu.\\
	\dicen{interwał (ang. \textit{interval})}-- przedział czasu, w którym odbywa się określona część treningu; zob. \textit{trening interwałowy}.\\
	\dicen{licznik (ang. \textit{tracker})}-- element aplikacji \textit{Iron Coach} służący do zbierania informacji o treningu, takich jak: czas, dystans, prędkość, tempo.\\
	\dicen{marszobieg (ang. \textit{walk-run})}-- trening łączący marsz i bieg w odpowiednich proporcjach czasowych, przykładowo: marszobieg złożony z pięciu serii (bez przerw między nimi), z których każda składa się z 2 minut biegu i następujących po nich 3 minut marszu.\\
	\dicen{odcinek (ang. \textit{lap})}-- jeden etap (część) całej trasy \mbox{(np. okrążenie).}\\
	\dicen{parametry osoby (ang. \textit{person parameters})}-- parametry takie jak: waga, wzrost, wiek (niezbędne do obliczenia zużycia energii podczas treningów).\\
	\dicen{plan treningowy (ang. \textit{train\-ing plan})}-- ustalony plan treningów na dany odcinek czasu (np. miesiąc, rok). Przykładowo: plan składający się z treningów interwałowych, w których w kolejnych dniach treningów wydłużany jest czas intensywnego wysiłku, skracany czas odpoczynku, a także np. wydłużany czas trwania całego treningu.\\
	\dicen{\mbox{rozgrzewka (ang. \textit{warming up})}}-- ćwiczenia wykonywane przed właściwym treningiem w celu rozgrzania mięśni i przygotowania organizmu do wysiłku fizycznego.\\
	\dicen{sportowiec (ang. \textit{sportsman})}-- osoba uprawiająca jakiś sport, \mbox{potencjalny użytkownik aplikacji.}\\
	\dicen{sprint (ang. \textit{sprint})}-- bardzo szybki bieg krótkodystansowy.\\
	\dicen{śledzenie (ang. \textit{tracking})}-- zapamiętywanie (na mapie) przebytej trasy.\\
	\dicen{tempo (ang. \textit{pace})}-- wielkość określająca czas biegu, w którym przebyta została ustalona jednostka drogi (np. w minutach na kilometr).
	\dicen{trening (ang. \textit{training})}-- wykonanie zaplanowanego zestawu następujących po sobie ćwiczeń fizycznych (biegów) w określonym czasie. \\
	\dicen{trening interwałowy (ang. \textit{interval training})}-- trening podzielony na przedziały czasu z intensywnym wysiłkiem przeplatane przedziałami z odpoczynkiem lub wysiłkiem o mniejszej intensywności.\\
	\dicen{trucht (ang. \textit{trot})}-- łagodna for\-ma treningu. Charakteryzuje się niewielką prędkością biegu, długim dystansem oraz długim czasem trwania.\\
  \end{description}
\end{multicols}
\end{document}
