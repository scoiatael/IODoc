\documentclass[11pt,wide]{mwart}

\usepackage[OT4,plmath]{polski}
\usepackage{amsmath,amssymb,amsfonts,amsthm,mathtools}
\usepackage{color}
\usepackage{fontspec}
\usepackage{listings,times}

\usepackage{bbm}
\usepackage[colorlinks=true, urlcolor=blue]{hyperref}
\usepackage{url}

\newcommand{\HRule}{\rule{\linewidth}{0.5mm}}

\newcommand{\dicen}[1]{
  \textbf{#1}
}
\usepackage{lipsum}

\usepackage{multicol}

\begin{document}
\begin{titlepage}
  \begin{center}

    \textsc{\LARGE Uniwersytet Wrocławski}\\[1.5cm]


    \vspace{3cm}

    % Author and supervisor
    \begin{minipage}{\textwidth}
      \begin{center} \Large
        Łukasz \textsc{Czapliński}
        Diana \textsc{Czepirska}
        Artur \textsc{Jarocki}
      \end{center}
    \end{minipage}

    \vspace{0.5cm}



    % Title
    \HRule \\[0.4cm]
    { \huge \bfseries <Dwulinijkowa>\\<nazwa projektu>  \\[0.4cm] }

    \HRule \\[1.5cm]

    % Upper part of the page. The '~' is needed because \\
    % only works if a paragraph has started.

    \vspace{1cm}

    \includegraphics[width=0.15\textwidth]{./non-starred.png}~\\[1cm]
    
    \vfill
    
    \textsc{\Large Projekt z inżynierii oprogramowania}\\[0.5cm]

    \vspace{1cm}

    % Bottom of the page
    {\large Wrocław 2013}

  \end{center}
\end{titlepage}
\section{\Large Słownik pojęć}
\begin{multicols}{2}
  % \dicen{pojecie} - definicja
Ogólne pojęcia:\\
	* \dicen{wysiłek fizyczny} - \lipsum[1-1]~\\ % <- zastąp definicjami, lipsum tworzy ten dziwny tekst
	* \dicen{trening} - \lipsum[1-1]~\\
	* \dicen{start, przerwa, koniec biegu} - \lipsum[1-1]~\\
	* \dicen{bieg}\\
	* \dicen{dystans}\\
	* \dicen{czas}\\
	* \dicen{prędkość}\\
	* \dicen{interwał}\\
	* \dicen{trasa}\\
	* \dicen{parametry osoby (waga, wzrost, wiek)}\\
	* \dicen{kcal spalone podczas treningu}\\
	* \dicen{okrążenie}\\
	* \dicen{automatyczny stop - poniżej danej prędkości licznik działa jak podczas przerwy (aż do osiągnięcia ponownie większej prędkości)}\\

Rodzaje treningów (biegów?):\\
	* \dicen{rozgrzewka}\\
	* \dicen{trucht}\\
	* \dicen{marszobieg}\\
	* \dicen{sprint}\\
	* \dicen{bieg górski}\\
	* \dicen{interwałowy (HIIT)}\\

\end{multicols}
\end{document}

