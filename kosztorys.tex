% Podstawowe definicje dla wszystkich dokumentów

\documentclass[11pt]{mwart}
\setlength{\textwidth}{83pt}

\usepackage[OT4,plmath]{polski}
\usepackage{amsmath,amssymb,amsfonts,amsthm,mathtools}
\usepackage{color}
\usepackage{fontspec}
\usepackage{listings,times}

\usepackage{bbm}
\usepackage[colorlinks=true, urlcolor=blue]{hyperref}
\usepackage{url}
\usepackage{graphicx}
\graphicspath{./images/}

\newcommand{\HRule}{\rule{\linewidth}{0.5mm}}

\newcommand{\term}[1]{
  \indent\textbf{#1}
  \vspace{5pt}
}

\usepackage{multicol}

\usepackage{lmodern} \normalfont
%\DeclareFontShape{EU1}{ptm}{bx}{n} { <-> ssub * cmr/bx/n }{}
%\DeclareFontShape{EU1}{ptm}{m}{sc} { <-> ssub * cmr/m/sc }{}

\usepackage{titletoc}

%\titlecontents{section}[3.8em]{}{\contentslabel{2.3em}}{\hspace*{-2.3em}}{\titlerule*[0.25pc]{ .}\contentspage}{}

\newcommand{\titlep}[1] {
  \begin{titlepage}
    \begin{center}
      \textsc{\LARGE Studencka Pracownia Inżynierii Oprogramowania}
      \textsc{\LARGE Instytut Informatyki Uniwersytetu Wrocławskiego}\\[1.5cm]


      \vspace{3cm}

      % Author and supervisor
      \begin{minipage}{\textwidth}
        \begin{center} \Large
          Łukasz \textsc{Czapliński},
          Diana \textsc{Czepirska},
          Artur \textsc{Jarocki}
        \end{center}
      \end{minipage}

      \vspace{0.5cm}



      % Title
      \HRule \\[0.4cm]
      { \Huge \bfseries Running\\free  \\[1cm] }

      \textsc{\Large #1}\\[0.5cm]

      \HRule \\[1.5cm]

      \vspace{1cm}

      \includegraphics[width=0.15\textwidth]{./non-starred.png}~\\[1cm]
      
      \vfill
      

      \vspace{1cm}

      % Bottom of the page
      {\large Wrocław 2013}

    \end{center}
  \end{titlepage}
  \clearpage
}


\begin{document}
\titlep{Kosztorys}{1.1}
\chist{1.0 & 2013-11-03 & Powstanie dokumentu & Łukasz Czapliński, Artur Jarocki \\
        1.1 & 2013-11-25 & Opis stadiów & Artur Jarocki \\
        1.11 & 2013-11-26 & Poprawki stylistyczne & Łukasz Czapliński}
\tableofcontents
\clearpage
\section{Wstęp}
Dokument ma na celu przybliżenie ostatecznego kosztu oraz czasu realizacji projektu Iron Coach. Do oszacowania została wykorzystana metoda COCOMO (ang. \textit{Constructive Cost Model}) w wersji podstawowej.

\section{Szacowanie kosztów metodą COCOMO}
\subsection{Szacowanie liczby linii kodu}
By oszacować liczbę osobogodzin potrzebnych do wykonania projektu, należy najpierw określić przybliżoną liczbę linii kodu gotowego produktu. Jest ona przedstawiana w KDSI (ang. \textit{(K) delivered source instructions}) gdzie 1 KDSI = 1000 linijek kodu. Na podstawie kodu projektu MyTracks składającego się z ok. 5000 linii szacujemy, że kod źródłowy aplikacji Iron Coach nie przekroczy 8000 linii, czyli 8 KDSI.

\subsection{Określenie złożoności}
Projekt został przydzielony do grupy projektów łatwych (ang. \textit{organic mode}), gdyż wielkość KDSI nie przekracza 50, będzie realizowany przez mały zespół oraz presja czasu jest mała. W tabeli określono stałe, potrzebne do określenia kolejno nakładu pracy, czasu potrzebnego do ukończenia projektu oraz sugerowanej liczby osób do optymalnej realizacji. 

\begin{equation}\label{Nakład pracy w osobomiesiącach}
E = a_b(KDSI)^{b_b} = 21.31
\end{equation}
\begin{equation}\label{Czas potrzebny do ukończenia projektu}
D = c_b(E)^{d_b} = 8.00
\end{equation}
\begin{equation}\label{Sugerowana liczba osób}
P = \frac{E}{D} = 2.63
\end{equation}

Na podstawie wyliczonych danych przyjęto, że realizacja zadania będzie powierzona zespołowi trzech ekspertów, szacując czas wykonania na 7 miesięcy. Rozkład kosztów powiązany z kolejnymi etapami produkcji przedstawia tabela 2.

\begin{table}[ht]
\begin{center}
\caption{Stałe łatewgo projektu COCOMO}
\begin{tabular}{| l | c |}
    \hline
		Stała & Wartość \\
	\hline
		a & 2.4 \\
	\hline
		b & 1.05 \\
	\hline
		c & 2.5 \\
	\hline
		d & 0.38 \\
    \hline

 \end{tabular}
 \end{center}
 \end{table}

  \begin{table}[ht]
  \begin{center}
  \caption{Koszty pracy}
  \begin{tabular}{| l | c | c | r |}
    \hline
      \multicolumn{1}{|c|}{Nazwa} & \multicolumn{1}{|c|}{Czas realizacji} & \multicolumn{1}{|c|}{Stawka za godzinę} & \multicolumn{1}{|c|}{Suma} \\
    \hline
      Analizowanie wymagań & 192h & 42 zł & 8064 zł \\
    \hline
      Tworzenie interfejsu użytkownika & 168h  & 34 zł & 5712 zł \\
    \hline
      Projektowanie bazy danych & 32h & 36 zł & 1152 zł \\
    \hline
      Pisanie aplikacji & 528h  & 40 zł & 21120 zł \\
    \hline
      Testowanie aplikacji & 136h & 27 zł & 3672 zł \\
    \hline
      Implementowanie nowych funkcji & 112h & 30 zł & 3360 zł \\
    \hline
      Tworzenie dokumentacji & 80h & 30 zł & 2400 zł \\
    \hline
      \multicolumn{3}{|l}{\textbf{Koszt całkowity}} & \textbf{45480 zł} \\
    \hline
  \end{tabular}
  \end{center}
  \end{table}
\end{document}
