\subsection{Znajdowanie potrzeb (ang. „Needfinding”)}

Termin: 29 października.

\subsection{Opracowanie prototypu rozwiązania}

Techniki:\begin{itemize}
\item przygotowanie prototypu papierowego,
\item opracowanie scenorysu (ang. storyboard),
\item ocenianie przez innych studentów z zastosowaniem heurystyk Nielsena poznanych wcześniej 
na zajęciach.
\end{itemize}
Termin: 26 listopada.

\subsection{Zbudowanie klikalnych makiet}

Propozycje narzędzi:\begin{itemize}
\item https://moqups.com (żeby zapisać makietę: http://d.pr/i/I0fH)
\item PowerPoint/Keynote (wyeksportowany do PDFa z aktywnymi łączami)
\item prosty HTML (stworzony np. na Boostrapie bądź w narzędziach typu http://www.divshot.com)
\item Visio/Fireworks/Flash
\end{itemize}
Termin: 10 grudnia.

\subsection{Napisanie silnika aplikacji}

Termin: 17 grudnia

\subsection{Iteracja na podstawie sprzężeń zwrotnych (ang. feedback)}

Na podstawie zebranych uwag wykonanie kolejnej iteracji rozwiązania, nie wszystkie elementy muszą być poprawione.\\

Termin: 14 stycznia.

\subsection{Testy z udziałem użytkowników}

Przetestowanie obu wersji. Zebranie notatek. Zaproponowanie kolejnej iteracji.\\

Termin: 28 stycznia.

\subsection{Ostateczne poprawki}

Termin: 14 lutego.
