\begin{enumerate}
  \item Znajdowanie potrzeb (ang. „Needfinding”)

    \term{Termin: 29 października.}

  \item Opracowanie prototypu rozwiązania

  Techniki:\begin{itemize}
  \item przygotowanie prototypu papierowego,
  \item opracowanie scenorysu (ang. storyboard),
  \item ocenianie przez innych studentów z zastosowaniem heurystyk Nielsena poznanych wcześniej 
  na zajęciach.
  \end{itemize}
  \term{Termin: 26 listopada.}

  \item Zbudowanie klikalnych makiet

  Propozycje narzędzi:\begin{itemize}
  \item https://moqups.com (żeby zapisać makietę: http://d.pr/i/I0fH)
  \item PowerPoint/Keynote (wyeksportowany do PDFa z aktywnymi łączami)
  \item prosty HTML (stworzony np. na Boostrapie bądź w narzędziach typu http://www.divshot.com)
  \item Visio/Fireworks/Flash
  \end{itemize}
  \term{Termin: 10 grudnia.}

  \item Iteracja na podstawie sprzężeń zwrotnych (ang. feedback)

  Na podstawie zebranych uwag wykonanie kolejnej iteracji rozwiązania, nie wszystkie elementy muszą być poprawione.

  \term{Termin: 14 stycznia.}

  \item Napisanie silnika aplikacji

    \term{Termin: 17 grudnia}


  \item Testy z udziałem użytkowników

  Przetestowanie obu wersji. Zebranie notatek. Zaproponowanie kolejnej iteracji.

  \term{Termin: 28 stycznia.}

  \item Ostateczne poprawki

    \term{Termin: 14 lutego.}
\end{enumerate}
