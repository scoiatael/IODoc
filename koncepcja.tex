% Podstawowe definicje dla wszystkich dokumentów

\documentclass[11pt]{mwart}
\setlength{\textwidth}{83pt}

\usepackage[OT4,plmath]{polski}
\usepackage{amsmath,amssymb,amsfonts,amsthm,mathtools}
\usepackage{color}
\usepackage{fontspec}
\usepackage{listings,times}

\usepackage{bbm}
\usepackage[colorlinks=true, urlcolor=blue]{hyperref}
\usepackage{url}
\usepackage{graphicx}
\graphicspath{./images/}

\newcommand{\HRule}{\rule{\linewidth}{0.5mm}}

\newcommand{\term}[1]{
  \indent\textbf{#1}
  \vspace{5pt}
}

\usepackage{multicol}

\usepackage{lmodern} \normalfont
%\DeclareFontShape{EU1}{ptm}{bx}{n} { <-> ssub * cmr/bx/n }{}
%\DeclareFontShape{EU1}{ptm}{m}{sc} { <-> ssub * cmr/m/sc }{}

\usepackage{titletoc}

%\titlecontents{section}[3.8em]{}{\contentslabel{2.3em}}{\hspace*{-2.3em}}{\titlerule*[0.25pc]{ .}\contentspage}{}

\newcommand{\titlep}[1] {
  \begin{titlepage}
    \begin{center}
      \textsc{\LARGE Studencka Pracownia Inżynierii Oprogramowania}
      \textsc{\LARGE Instytut Informatyki Uniwersytetu Wrocławskiego}\\[1.5cm]


      \vspace{3cm}

      % Author and supervisor
      \begin{minipage}{\textwidth}
        \begin{center} \Large
          Łukasz \textsc{Czapliński},
          Diana \textsc{Czepirska},
          Artur \textsc{Jarocki}
        \end{center}
      \end{minipage}

      \vspace{0.5cm}



      % Title
      \HRule \\[0.4cm]
      { \Huge \bfseries Running\\free  \\[1cm] }

      \textsc{\Large #1}\\[0.5cm]

      \HRule \\[1.5cm]

      \vspace{1cm}

      \includegraphics[width=0.15\textwidth]{./non-starred.png}~\\[1cm]
      
      \vfill
      

      \vspace{1cm}

      % Bottom of the page
      {\large Wrocław 2013}

    \end{center}
  \end{titlepage}
  \clearpage
}

\begin{document}
\titlep{Koncepcja}{1.1}
\chist{1.0 & 2013-10-26 & Powstanie dokumentu & Łukasz Czapliński\\}
\tableofcontents
\clearpage
\section{Wprowadzenie}
\subsection{Cel dokumentu}
\subsection{Cel projektu}
Projekt ,,Running free`` ma na celu stworzenie aplikacji na urządzenia mobilne, której użytkownikami mają docelowo być osoby uprawiające sport. Ma ona pomagać im w treningach -- ułatwiając ich planowanie, mierząc czas i osiągnięcia oraz motywując. 
Program ma działać w dwóch trybach: treningu oraz podsumowania.

W pierwszym z nich użytkownik będzie miał możliwość rozpoczęcia (kontynuowania) treningu. Wiąże się to z wyborem trybu (np trening kondycyjny lub interwałowy).
W drugim trybie korzystający z aplikacji będzie mógł obejrzeć swoje dotychczasowe osiągnięcia i statystyki: ile ćwiczył, jak często. 
\section{Charakterystyka docelowego użytkownika}
\section{Prognozy realizacji projektu i odbioru apliacji}
\section{Podstawowe cechu funkcjonalne aplikacji}
\section{Wymagania niefunkcjonalne stawiane projektowi i aplikacji}
\section{Dokumentacja projektu}
\section{Słownik}
\begin{thebibliography}{99}
  \bibitem{Sł} Łukasz Czapliński, Diana Czepirska, Artur Jarocki: {\it Dokumentacja projektu Iron Coach. Słownik pojęć}, Wrocław, SPIO IIUWr 2013
\end{thebibliography}
\end{document}
