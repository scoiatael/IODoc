% Podstawowe definicje dla wszystkich dokumentów

\documentclass[11pt]{mwart}
\setlength{\textwidth}{83pt}

\usepackage[OT4,plmath]{polski}
\usepackage{amsmath,amssymb,amsfonts,amsthm,mathtools}
\usepackage{color}
\usepackage{fontspec}
\usepackage{listings,times}

\usepackage{bbm}
\usepackage[colorlinks=true, urlcolor=blue]{hyperref}
\usepackage{url}
\usepackage{graphicx}
\graphicspath{./images/}

\newcommand{\HRule}{\rule{\linewidth}{0.5mm}}

\newcommand{\term}[1]{
  \indent\textbf{#1}
  \vspace{5pt}
}

\usepackage{multicol}

\usepackage{lmodern} \normalfont
%\DeclareFontShape{EU1}{ptm}{bx}{n} { <-> ssub * cmr/bx/n }{}
%\DeclareFontShape{EU1}{ptm}{m}{sc} { <-> ssub * cmr/m/sc }{}

\usepackage{titletoc}

%\titlecontents{section}[3.8em]{}{\contentslabel{2.3em}}{\hspace*{-2.3em}}{\titlerule*[0.25pc]{ .}\contentspage}{}

\newcommand{\titlep}[1] {
  \begin{titlepage}
    \begin{center}
      \textsc{\LARGE Studencka Pracownia Inżynierii Oprogramowania}
      \textsc{\LARGE Instytut Informatyki Uniwersytetu Wrocławskiego}\\[1.5cm]


      \vspace{3cm}

      % Author and supervisor
      \begin{minipage}{\textwidth}
        \begin{center} \Large
          Łukasz \textsc{Czapliński},
          Diana \textsc{Czepirska},
          Artur \textsc{Jarocki}
        \end{center}
      \end{minipage}

      \vspace{0.5cm}



      % Title
      \HRule \\[0.4cm]
      { \Huge \bfseries Running\\free  \\[1cm] }

      \textsc{\Large #1}\\[0.5cm]

      \HRule \\[1.5cm]

      \vspace{1cm}

      \includegraphics[width=0.15\textwidth]{./non-starred.png}~\\[1cm]
      
      \vfill
      

      \vspace{1cm}

      % Bottom of the page
      {\large Wrocław 2013}

    \end{center}
  \end{titlepage}
  \clearpage
}

\begin{document}
\titlep{Założenia ogólne}{1.1}
\chist{1.0 & 2013-10-26 & Powstanie dokumentu & Łukasz Czapliński\\
        1.1 & 2013-11-28 & Rozwinięcie sekcji dot użytkownika, prognoz i wymagań & Łukasz Czapliński\\}
\tableofcontents
\clearpage
\section{Wprowadzenie}
\subsection{Cel dokumentu}
\noindent Poniższy dokument ma na celu przedstawienie pokrótce projektu ,,Iron Coach'' i jego celu -- aplikacji mobilnej. Jest to zarys bardziej szczegółowych specyfikacji rozwiniętych w pozostałej części dokumentacji projektu. Przedstawiono cele projektu, typowego użytkownika aplikacji oraz jego wymagania wobec niej, jak również prognozy dotyczące realizacji i odbioru projektu. Następnie wypisano wymagania stawiane projektowi. Przedstawiono również, jakie pomoce dla użytkownika (dokumentacja) będą dołączone do aplikacji.
\subsection{Cel i nazwa projektu}
\noindent Projekt ,,Iron Coach'' ma na celu napisanie aplikacji na urządzenia mobilne, której użytkownikami mają docelowo być osoby uprawiające różnego rodzaje sporty. Ma ona pomagać im w treningach, ułatwiając ich planowanie oraz motywując do osiągania lepszych wyników. \\
Potrzeby użytkowników, które ma zaspokajać ta aplikacja, można opisać bardzo prosto:\\ 
\indent ,,Wiele osób uprawiających sporty odczuwa brak różnorakich udogodnień (np. mierzenie dystansu dla biegaczy, znajdywanie tras rowerowych dla cyklistów, drogi widokowej dla turystów, układanie planu treningów dla początkujących), które można rozwiązać aplikacją mobilną.''\cite{Ba}
Rozwiązanie właśnie tych problemów jest celem projektu ,,Iron Coach''. 
Nazwa projektu wzięła się z prostego zapożyczenia: tak jak ,,Iron Man©'' jest geniuszem w swojej dziedzinie, tak ,,Iron Coach'' będzie idealnym trenerem. Planowane jest opracowanie wersji aplikacji w języku angielskim w celu udostępnienia jej nie tylko dla polskim użytkownikom.
\section{Charakterystyka docelowego użytkownika}
\noindent Wedle danych GUS-u(\cite{Gu}), ponad 37\% mieszkańców Polski uprawia różne sporty. Wiekszość z nich to amatorzy o średnich dochodach. Właśnie oni mają stać się użytkownikami aplikacji ,,Iron Coach''. Badanie wykonane w fazie wstępnej projektu (\cite{Ba}) pokazują, że wielu amatorów sportu odczuwa problemy, które mogą być polem dla wielu innowacji.
\\Podsumowując, docelowy użytkownik aplikacji ,,Iron Coach'':
\begin{itemize}
  \item posiada smartfona,
  \item uprawia sport (amatorsko lub profesjonalnie),
  \item ma niezaspokojne problemy związane ze swoim sportem, na przykład:
    \begin{itemize}
      \item nie potrafi sam ułożyć planu treningowego,
      \item chciałby śledzić swoje osiągnięcia,
      \item potrzebuje motywacji, by trenować regularnie.
    \end{itemize}
\end{itemize}
\section{Prognozy realizacji projektu i odbioru aplikacji}
\subsection{Realizacja projektu}
\noindent Prognozy dotyczące projektu i aplikacji można tworzyć na podstawie wielu obserwacji. Badania prowadzone w fazie wstępnej projektu ,,Iron Coach'' \cite{Ba} dotyczyły rynku aplikacji skierowanych do sportowców. Wynikało z nich, że większość dostępnych na rynku aplikacji jest tworzona amatorsko i w związku z tym obciążona dużą ilością błędów związanych z brakiem kontroli jakości. Jednocześnie nie są one trudne do napisania (p. \ref{pfunk}). Można więc twierdzić, iż projekt ,,Iron Coach'' ma wysokie szanse stworzenia aplikacji wolnej od błędów w określonym harmonogramem terminie (ok. 7 miesięcy, patrz \cite{Ha}). 
\subsection{Odbiór aplikacji}
\noindent Badano również rynek gadżetów dla amatorów sportów. Z dynamicznego rozwoju tego rynku można wnioskować, że wielu ludzi wydaje pieniądze na pomoce, jak np: ciśnieniomierz, gps, aparatu do pomiaru wyników i układania planów treningowych. Wszystkie te gadżety mogą być zastąpione przez jedną aplikację mobilną. Dodatkowa zintegrowanie jej z serwisem internetowym i opracowanie wersji na komputery osobiste daje pole do rozwoju. Można więc śmiało wnioskować, że aplikacja ,,Iron Coach'' znajdzie odbiorców w dużej liczbie. \\
Podsumowując, wysokie szanse realizacji i rozwoju połączone z bogatą rzeszą odbiorców pozwalają dobrze rokować powodzeniu projektu ,,Iron Coach''.
\clearpage
\section{Wymagania stawiane projektowi i aplikacji}
\subsection{Podstawowe cechu funkcjonalne aplikacji}
\label{pfunk}
\noindent Aplikacja ,,Iron Coach'' powinna umożliwiać użytkownikowi:
\begin{itemize}
  \item planowanie harmonogramu treningów,
  \item łatwe motywowanie do regularnych treningów,
  \item śledzenie regularności i wyników treningów,
  \item łatwe rozszerzanie o moduły treningowe (pobierane ze sklepu lub internetu), z których każdy powinien pozwalać na:
    \begin{itemize}
      \item wykonywanie wybranego przez użytkownika treningu,
      \item dopasowanie typu treningu do użytkownika,
      \item przyznawanie nagród za wybitne osiągnięcia,
      \item ocenę każdego z treningów,
      \item kalibrację urządzeń stosowanych do pomiaru treningu.
    \end{itemize}
\end{itemize}
\subsection{Wymagania niefunkcjonalne}
\noindent Aplikacja ,,Iron Coach'' powinna:
\begin{itemize}
  \item posiadać interfejs o następujących cechach:
    \begin{itemize}
      \item estetyczny,
      \item intuicyjny, 
      \item łatwy w obsłudze (oferujący pomoc kontekstową), 
    \end{itemize}
  \item dobrze reagować na polecenia użytkownika (reakcja do 0.5 sek),
  \item dobrze oceniać trening użytkownika,
  \item być niezawodna (mniej niż jeden błąd niekrytyczny na 2h treningów, brak błędów krytycznych),
  \item być dostępna na co najmniej dwóch głównych platformach mobilnych: android i iOS,
  \item być zgodna z wymogami i zaleceniami tworzenia aplikacji  na każdą platformę, na którą będzie działać (dopuszczalne jest, by wersje na osobne platformy posiadały inne interfejsy, każdy spełniający odpowiednie wymogi).
\end{itemize}
\vspace{2pt}
Projekt ,,Iron Coach'' powinien:
\begin{itemize}
  \item na bieżąco informować o postępach w tworzeniu aplikacji,
  \item nie ukrywać problemów w tworzeniu aplikacji,
  \item spełniać wymogi zarządzania projektem PRINCE2(\cite{Pr}).
\end{itemize}
\section{Dokumentacja projektu}
\noindent Częścią programu będzie pomoc kontekstowa, która powinna zaspokajać podstawowe potrzeby normalnego użytkownika. Potrzebę sięgnięcia do dokumentacji projektowej i użytkowej powinni odczuwać użytkownicy mający nietypowe potrzeby wobec aplikacji. Z tego powodu będzie ona udostępniona razem z kodem źródłowym aplikacji.
\section{Słownik}
\noindent Słownik specyficznych terminów, wymagających objaśnienia, użytych w tym i pozostałych dokumentach znajduje się w~\cite{Sł}.
\begin{thebibliography}{99}
  \bibitem{Ba} Przemysław Sierociński: {\it Iron Coach. Needfinding}. Wrocław, SPKCK IIUWr 2013
  \bibitem{Gu} GUS, Departament Badań Społecznych i Warunków Życia, {\it KULTURA FIZYCZNA W POLSCE W LATACH 2008-2010}. Warszawa, 2011
  \bibitem{Sł} Łukasz Czapliński, Diana Czepirska, Artur Jarocki: {\it Dokumentacja projektu Iron Coach. Słownik pojęć}. Wrocław, SPIO IIUWr 2013
  \bibitem{Ha} Łukasz Czapliński, Diana Czepirska, Artur Jarocki: {\it Dokumentacja projektu Iron Coach. Harmonogram}. Wrocław, SPIO IIUWr 2013
  \bibitem{Pr} PRINCE2 Official website, \url{http://www.prince-officialsite.com/}.
\end{thebibliography}
\end{document}
