% Podstawowe definicje dla wszystkich dokumentów

\documentclass[11pt]{mwart}
\setlength{\textwidth}{83pt}

\usepackage[OT4,plmath]{polski}
\usepackage{amsmath,amssymb,amsfonts,amsthm,mathtools}
\usepackage{color}
\usepackage{fontspec}
\usepackage{listings,times}

\usepackage{bbm}
\usepackage[colorlinks=true, urlcolor=blue]{hyperref}
\usepackage{url}
\usepackage{graphicx}
\graphicspath{./images/}

\newcommand{\HRule}{\rule{\linewidth}{0.5mm}}

\newcommand{\term}[1]{
  \indent\textbf{#1}
  \vspace{5pt}
}

\usepackage{multicol}

\usepackage{lmodern} \normalfont
%\DeclareFontShape{EU1}{ptm}{bx}{n} { <-> ssub * cmr/bx/n }{}
%\DeclareFontShape{EU1}{ptm}{m}{sc} { <-> ssub * cmr/m/sc }{}

\usepackage{titletoc}

%\titlecontents{section}[3.8em]{}{\contentslabel{2.3em}}{\hspace*{-2.3em}}{\titlerule*[0.25pc]{ .}\contentspage}{}

\newcommand{\titlep}[1] {
  \begin{titlepage}
    \begin{center}
      \textsc{\LARGE Studencka Pracownia Inżynierii Oprogramowania}
      \textsc{\LARGE Instytut Informatyki Uniwersytetu Wrocławskiego}\\[1.5cm]


      \vspace{3cm}

      % Author and supervisor
      \begin{minipage}{\textwidth}
        \begin{center} \Large
          Łukasz \textsc{Czapliński},
          Diana \textsc{Czepirska},
          Artur \textsc{Jarocki}
        \end{center}
      \end{minipage}

      \vspace{0.5cm}



      % Title
      \HRule \\[0.4cm]
      { \Huge \bfseries Running\\free  \\[1cm] }

      \textsc{\Large #1}\\[0.5cm]

      \HRule \\[1.5cm]

      \vspace{1cm}

      \includegraphics[width=0.15\textwidth]{./non-starred.png}~\\[1cm]
      
      \vfill
      

      \vspace{1cm}

      % Bottom of the page
      {\large Wrocław 2013}

    \end{center}
  \end{titlepage}
  \clearpage
}

\begin{document}
\titlep{Założenia ogólne}{1.1}
\chist{1.0 & 2013-10-26 & Powstanie dokumentu & Łukasz Czapliński\\
        1.1 & 2013-11-28 & Rozwinięcie sekcji dot użytkownika, prognoz i wymagań & Łukasz Czapliński\\}
\tableofcontents
\clearpage
\section{Wprowadzenie}
\subsection{Cel dokumentu}
Poniższy dokument ma na celu przedstawienie pokrótce projektu Iron Coach i jego celu - aplikacji mobilnej. Jest to zarys bardziej szczegółowych specyfikacji rozwiniętych w reszcie dokumentacji projektu. Przedstawione są cele projektu, typowy użytkownik aplikacji oraz jego wymagania wobec niej, jak również prognozy dotyczące realizacji i odbioru. Następnie wypisano wymagania stawiane projektowi. Przedstawiono również jakie pomoce dla użytkownika (dokumentacja) będzie dołączona do apliacji.
\subsection{Cel i nazwa projektu}
Projekt ,,Iron Coach`` ma na celu napisanie aplikacji na urządzenia mobilne, której użytkownikami mają docelowo być osoby uprawiające różnego rodzaje sporty. Ma ona pomagać im w treningach -- ułatwiając ich planowanie oraz motywując do osiągania lepszych wyników. \\
Potrzeby użytkowników, jakie ma zaspokajać ta aplikacja można opisać bardzo prosto:\\ 
,,Wiele osób uprawiających sporty odczuwa różnorakie problemy (np mierzenie dystansu dla biegaczy, znajdywanie tras rowerowych dla cyklistów, drogi widokowej dla turystów, układanie planu treningów dla początkujących), które można rozwiązać aplikacją mobilną.``
Rozwiązanie właśnie tych problemów jest celem projektu ,,Iron Coach``. 
Nazwa projektu wzięła się z prostego zapożyczenia: tak jak ,,Iron man`` jest geniuszem w swojej dziedzinie, tak ,,Iron Coach`` będzie idealnym trenerem. Planowane jest stworzenie aplikacji w języku angielskim, w celu udostępnienia jej nie tylko dla polskich użytkowników.
\section{Charakterystyka docelowego użytkownika}
\section{Prognozy realizacji projektu i odbioru apliacji}
\section{Podstawowe cechu funkcjonalne aplikacji}
\section{Wymagania niefunkcjonalne stawiane projektowi i aplikacji}
\section{Dokumentacja projektu}
Częścią programu będzie pomoc kontekstowa, która powinna zaspokajać podstawowe potrzeby normalnego użytkownika. Potrzebę sięgnięcia do dokumentacji projektowej i użytkowej powinni odczuwać użytkownicy mający nietypowe potrzeby wobec aplikacji. Dlatego będzie ona udostępniona razem z kodem źródłowym aplikacji.
\section{Słownik}
Słownik specyficznych terminów, wymagających objaśnienia, użytych w tym i pozostałych dokumentach znajduje się w~\cite{Sł}.
\begin{thebibliography}{99}
  \bibitem{Sł} Łukasz Czapliński, Diana Czepirska, Artur Jarocki: {\it Dokumentacja projektu Iron Coach. Słownik pojęć}, Wrocław, SPIO IIUWr 2013
\end{thebibliography}
\end{document}
