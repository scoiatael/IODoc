% \dicen{pojecie} - definicja
	\noindent\dicen{automatyczna pauza (ang. \textit{auto pause})}-- działanie licznika polegające na tym, że poniżej danej prędkości zaczyna on działać jak podczas przerwy (tzn. przestaje naliczać czas i dystans) aż do momentu ponownego osiągnięcia większej prędkości.\\
	\dicen{bieg górski (ang. \textit{mountain running})}-- intensywny trening polegający na biegu w terenie, na którym suma przewyższeń stanowi minimum kilka procent (np. 5\%) całej trasy.\\
	\dicen{energia (ang. \textit{energy})}-- oszacowane zużycie energii (liczba spalonych kalorii) podczas treningu.\\
	\dicen{HIIT (z ang. \textit{High Intensity Interval Training} -- trening interwałowy o wysokiej intensywności)}-- rodzaj treningu interwałowego, który składa się z krótkich okresów bardzo intensywnego wysiłku przeplatanych krótkimi okresami umiarkowanego wysiłku, przykładowo: 30 sekund sprintu na przemian z 15 sekundami truchtu.\\
	\dicen{interwał (ang. \textit{interval})}-- przedział czasu.\\
	\dicen{licznik}-- element aplikacji służący do zbierania informacji o treningu, takich jak: czas, dystans, prędkość, tempo.\\
	\dicen{marszobieg (ang. \textit{walk-run})}-- trening łączący marsz i bieg w odpowiednich proporcjach czasowych, przykładowo: marszobieg złożony z pięciu serii (bez przerw między nimi), z których każda składa się z 2 minut biegu i następujących po nich 3 minut marszu.\\
	\dicen{odcinek (ang. \textit{lap})}-- jednen etap/jedna część całej trasy (np. okrążenie).\\
	\dicen{osiągnięcia (ang. \textit{achievments})}-- odznaczenia, które użytkownicy mogą uzyskać za osiąganie wyznaczonych celów. System osiągnięć służy do motywacji użytkówników i doprowadzenia do rywalizacji między nimi.\\
	\dicen{parametry osoby}-- parametry takie jak: waga, wzrost, wiek -- niezbędne do obliczenia zużycia energii podczas treningów.\\
	\dicen{plan treningowy}-- ustalony plan treningów na dany odcinek czasu (składający się np. z treningów interwałowych, w których w kolejnych dniach treningów wydłużany jest czas intensywnego wysiłku, skracany czas odpoczynku, a także np. wydłużany czas trwania całego treningu).\\
	\dicen{rozgrzewka (ang. \textit{warming up})}-- ćwiczenia wykonywane przed właściwym treningiem w celu rozgrzania mięśni i przygotowania organizmu do wysiłku fizycznego.\\
	\dicen{sprint}-- bardzo szybki bieg krótkodystansowy.\\
	\dicen{śledzenie (ang. \textit{tracking})}-- zapamiętywanie (na mapie) przebiegniętej trasy\\
	\dicen{tempo (ang. \textit{pace})}-- czas treningu potrzebny do przebiegnięcia danej jednostki odległości (np. w min/km).\\
	\dicen{trening (ang. \textit{training})}-- wykonanie zaplanowanego zestawu następujących po sobie ćwiczeń fizycznych (biegów) w określonym czasie. \\
	\dicen{trening interwałowy (in. trening przedziałowy, ang. \textit{interval training})}-- trening składający się z serii intensywnego wysiłku przeplatanych odpoczynkiem lub wysiłkiem o mniejszej intensywności.\\
	\dicen{trucht (ang. \textit{trot})}-- łagodna forma treningu. Charakteryzuje się niewielką prędkością biegu, długim dystansem oraz długim czasem trwania.\\
