% \dicen{pojecie} - definicja
	\noindent\dicen{automatyczna pauza \textit{(ang. auto pause)}} - działanie licznika polegające na tym, że poniżej danej prędkości zaczyna on działać jak podczas przerwy (tzn. przestaje naliczać czas i dystans) aż do momentu ponownego osiągnięcia większej prędkości.\\
	\dicen{bieg górski \textit{(ang. mountain running)}} - intensywny trening polegający na biegu w terenie, na którym suma przewyższeń stanowi minimum kilka procent (np. 5\%) całej trasy.\\
	\dicen{energia \textit{(ang. energy)}} - oszacowane zużycie energii (liczba spalonych kalorii) podczas treningu.\\
	\dicen{HIIT \textit{(z ang. High Intensity Interval Training)}} - trening interwałowy o wysokiej intensywności)- rodzaj treningu interwałowego, który składa się z krótkich okresów bardzo intensywnego wysiłku przeplatanych krótkimi okresami umiarkowanego wysiłku, przykładowo: 30 sekund sprintu na przemian z 15 sekundami truchtu.\\
	\dicen{interwał \textit{(ang. interval)}} - przedział czasu.\\
	\dicen{licznik} - element aplikacji służący do zbierania informacji dot. treningu, takich jak: czas, dystans, prędkość, tempo.\\
	\dicen{marszobieg \textit{(ang. walk-run)}} - trening łączący marsz i bieg w odpowiednich proporcjach czasowych, przykładowo: marszobieg złożony z pięciu serii (bez przerw między nimi), z których każda składa się z 2 minut biegu i następujących po nich 3 minut marszu.\\
	\dicen{odcinek \textit{(ang. lap)}} - jednen etap/jedna część całej trasy (np. okrążenie).\\
	\dicen{osiągnięcia \textit{(ang. achievments)}} - odznaczenia, które użytkownicy mogą uzyskać za osiąganie wyznaczonych celów. System osiągnięć służy do motywacji użytkówników i doprowadzenia do rywalizacji między nimi.\\
	\dicen{parametry osoby} - parametry takie jak: waga, wzrost, wiek - niezbędne do obliczenia zużycia energii podczas treningów.\\
	\dicen{plan treningowy} - ustalony plan treningów na dany odcinek czasu (składający się np. z treningów interwałowych, w których w kolejnych dniach treningów wydłużany jest czas intensywnego wysiłku, skracany czas odpoczynku, a także np. wydłużany czas trwania całego treningu).\\
	\dicen{rozgrzewka \textit{(ang. warming up)}} - ćwiczenia wykonywane przed właściwym treningiem w celu rozgrzania mięśni i przygotowania organizmu do wysiłku fizycznego.\\
	\dicen{sprint} - bardzo szybki bieg krótkodystansowy.\\
	\dicen{śledzenie \textit{(ang. tracking)}} - zapamiętywanie (na mapie) przebiegniętej trasy\\
	\dicen{tempo \textit{(ang. pace)}} - czas treningu potrzebny do przebiegnięcia danej jednostki odległości (np. w min/km).\\
	\dicen{trening \textit{(ang. training)}} - wykonanie zaplanowanego zestawu następujących po sobie ćwiczeń fizycznych (biegów) w określonym czasie. \\
	\dicen{trening interwałowy \textit{(in. trening przedziałowy, ang. interval training)}} - trening składający się z serii intensywnego wysiłku przeplatanych odpoczynkiem lub wysiłkiem o mniejszej intensywności.\\
	\dicen{trucht \textit{(ang. trot)}} - łagodna forma treningu. Charakteryzuje się niewielką prędkością biegu, długim dystansem oraz długim czasem trwania\\
